\documentclass[10pt]{beamer}

% -------- Core packages (consistent with article preamble) ----------
\usepackage[utf8]{inputenc}
%\usepackage[OT1]{fontenc}
\usepackage{helvet}
\renewcommand{\familydefault}{\sfdefault}
\usepackage{listings}



% -------- Beamer base theme (minimal chrome) ----------
\mode<presentation>{
  \usetheme{default}
  \useinnertheme{rounded}
  \useoutertheme{infolines} % slim footer line
  \setbeamertemplate{navigation symbols}{} % no nav icons
}


% -------- Color palette (copied from article) ----------
\definecolor{C1}{RGB}{141,125,158} % purple
\definecolor{C2}{RGB}{163,156,147} % yellowish
\definecolor{C3}{RGB}{99,151,153}  % green
\definecolor{C4}{RGB}{195,106,99}  % red
\definecolor{C5}{RGB}{124,132,128}
\definecolor{C6}{RGB}{67,69,75}

\definecolor{Gr}{HTML}{377D71}
\definecolor{Pu}{HTML}{A459D1}
\definecolor{Bl}{HTML}{4D455D}
\definecolor{Te}{HTML}{C1ECE4}
\definecolor{Or}{HTML}{EF6262}
\definecolor{Am}{HTML}{F3AA60}
\definecolor{Co}{HTML}{3C486B}
\definecolor{Wh}{HTML}{FEFBF6}
\definecolor{Ye}{HTML}{FFE196}
\definecolor{Re}{HTML}{E96479}
\definecolor{Pi}{HTML}{FFD0D0}
\definecolor{Rp}{HTML}{FF9EAA}
\definecolor{Wg}{HTML}{EEF3D2}

% -------- Beamer color + font mapping (subtle, article-like) ----------
\setbeamercolor{background canvas}{bg=Wh}
\setbeamercolor{normal text}{fg=black}
\setbeamercolor{structure}{fg=C5} % itemize bullets, section titles
\setbeamercolor{frametitle}{fg=C6,bg=}
\setbeamercolor{title}{fg=C6}
\setbeamercolor{subtitle}{fg=C5}
\setbeamercolor{section in toc}{fg=C6}
\setbeamercolor{itemize item}{fg=black}
\setbeamercolor{itemize subitem}{fg=C5}
\setbeamercolor{block title}{fg=Wh,bg=C6}
\setbeamercolor{block body}{bg=Wg!40,fg=Bl}
\setbeamercolor{alerted text}{fg=C4}
\setbeamercolor{example text}{fg=C3}

\setbeamertemplate{itemize item}{\color{black}$\blacktriangleright$} % first-level
\setbeamertemplate{itemize subitem}{\color{black}$\bullet$}           % second-level
\setbeamertemplate{itemize subsubitem}{\tiny$\blacksquare$}    
% Arabic with a closing parenthesis: 1) 2) 3)
\setbeamertemplate{enumerate item}{\arabic{enumi})}
\setbeamertemplate{enumerate subitem}{(\alph{enumii})}   % (a) (b) ...
\setbeamertemplate{enumerate subsubitem}{\roman{enumiii}.} % i. ii. iii.


% Change section numbering format
\setbeamertemplate{section in toc}[sections numbered]
\setbeamertemplate{subsection in toc}[subsections numbered]




% ...

 \title{Unit 1}
 \subtitle{Basic Probability and simulation}
 \author{Ethan Levien}
 \institute{Math 50}
 \date{\today}



\begin{document}

% Title slide
\begin{frame}
  \titlepage
\end{frame}

% Outline
\begin{frame}{Outline}
  \tableofcontents
\end{frame}

% Section: Introduction
\section{Introduction}
\begin{frame}{What is a Model?}
  \begin{itemize}
    \item Models are simplified representations of the world.
    \item Examples of models include
      \begin{itemize}
        \item Astrology. this is a model of human behavior
        \item Newton’s laws
      \end{itemize}
    \item We focus on \textbf{statistical models} (also known as \textbf{probability models}): These are models where the variables can not be predicted exactly. 
    \item {\bf Goal of unit 1}: develop the language and notation for describing randomness mathematically. 
  \end{itemize}
\end{frame}

\section{Random variables and probability distributions}
\begin{frame}{Example: The Bernoulli random variable}

How do we model a binary event such as: The flip of a coin, whether it rains etc. 

We introduce a random variable called a Bernoulli random variable 
\begin{equation*}
X \sim {\rm Bernoulli}(q)
\end{equation*}
This reads as ``X is a Bernoulli random variable'', which means
\begin{enumerate}
\item  {\bf Sample space:} $X$ is either one or zero 
\item {\bf Probability distribution} $X=1$ is $q$, which we write as
\begin{equation*}
P(X = 1) = q \text{  or  } P_X(1) = q
\end{equation*}
Note that $P_X(1) + P_X(0) = 1$, hence $P_X(0) = 1-q$. We could also write
\begin{equation*}
P(X = x)  = \left\{\begin{array}{lr}
q & x =1\\
1-q & x =0
 \end{array} \right.
\end{equation*}
\end{enumerate}


\end{frame}


\begin{frame}[fragile]{Generating random variables with Python}
\begin{lstlisting}[language=Python]
import numpy as np

# Bernoulli with probability p = 0.5
p = 0.5
samples = np.random.binomial(1, p, 100)
print(samples)
\end{lstlisting}
\end{frame}


\begin{frame}
\frametitle{Tips for using ChatGPT to help with coding} 
\begin{itemize}
\item Include ``In Python using numpy'' in the prompt
\item Ask it to break down the syntax if you don't understand what it returns (for simple things)
\item If you know another language, try writing the code in that language and ask it to convert it to Python. This is a code way to see how syntax you know maps to python. 
\end{itemize}

\end{frame}


\begin{frame}
\frametitle{General definitions} 

\begin{itemize}
\item The {\bf Sample space}  $S$ is the set of all outcomes of an experiment, {\bf events} are subsets of the sample space. 
\item We write the probabilities as 
\begin{equation}
P(X = x) = P_X(x),\quad x \in S
\end{equation} 
Sometimes we use ${\mathbb P}$ instead of $P$. 
\item If the distribution has a name, then we write 
\begin{equation}
X \sim {\rm NameOfDistribution}(\text{parameters})
\end{equation}
 where {\bf parameters} are numbers that effect the probabilities. E.g. $q$ for the Bernoulli distribution. 
 \item If we have multiple random variables, $X,Y,Z,...$ we write 
 \begin{equation}
 P(X=x,Y=y) = P_{X,Y}(x,y)
 \end{equation}
  to represent the probability that $X=x$ and $Y=y$ at the same time. 
\end{itemize}

\end{frame}





\begin{frame}
\frametitle{Examples}
\begin{itemize}
\item Let $X_1,X_2$ bet the outcome of two independent coin flips. What is the sample space? 
\item A coin is flipped and then a dice is rolled. What is the sample space? If they are fair, what is the probability distribution? 
\item A survey is given to 3 students with two yes know questions. What is the sample space? 
\end{itemize}
\end{frame}


\begin{frame}
\frametitle{Properties of probability distributions}
\begin{enumerate}
\item Probabilities of disjoint events add together. For example, the probability of rolling $1$ of $2$ is $2/6$. 
\item Probability
\item Probabilities multiple. 
\end{enumerate}
\end{frame}

\begin{frame}
\frametitle{How likely are there aliens?}

\end{frame}

\begin{frame}
\frametitle{Probability distributions from data}


\end{frame}


\section{Independence, conditioning and marginalization}
\begin{frame}{Conditional probability}
Conditional probabilities are probabilities we obtain when we restrict ourselves to certain events in the sample space.



\end{frame}

\begin{frame}{Conditional probability examples}



\end{frame}


\begin{frame}{Marginal probabilities}

\end{frame}




\end{document}