\documentclass[10pt]{beamer}

% -------- Core packages (consistent with article preamble) ----------
\usepackage[utf8]{inputenc}
%\usepackage[OT1]{fontenc}
\usepackage{helvet}
\renewcommand{\familydefault}{\sfdefault}
\usepackage{listings}



% -------- Beamer base theme (minimal chrome) ----------
\mode<presentation>{
  \usetheme{default}
  \useinnertheme{rounded}
  \useoutertheme{infolines} % slim footer line
  \setbeamertemplate{navigation symbols}{} % no nav icons
}


% -------- Color palette (copied from article) ----------
\definecolor{C1}{RGB}{141,125,158} % purple
\definecolor{C2}{RGB}{163,156,147} % yellowish
\definecolor{C3}{RGB}{99,151,153}  % green
\definecolor{C4}{RGB}{195,106,99}  % red
\definecolor{C5}{RGB}{124,132,128}
\definecolor{C6}{RGB}{67,69,75}

\definecolor{Gr}{HTML}{377D71}
\definecolor{Pu}{HTML}{A459D1}
\definecolor{Bl}{HTML}{4D455D}
\definecolor{Te}{HTML}{C1ECE4}
\definecolor{Or}{HTML}{EF6262}
\definecolor{Am}{HTML}{F3AA60}
\definecolor{Co}{HTML}{3C486B}
\definecolor{Wh}{HTML}{FEFBF6}
\definecolor{Ye}{HTML}{FFE196}
\definecolor{Re}{HTML}{E96479}
\definecolor{Pi}{HTML}{FFD0D0}
\definecolor{Rp}{HTML}{FF9EAA}
\definecolor{Wg}{HTML}{EEF3D2}

% -------- Beamer color + font mapping (subtle, article-like) ----------
\setbeamercolor{background canvas}{bg=Wh}
\setbeamercolor{normal text}{fg=black}
\setbeamercolor{structure}{fg=C5} % itemize bullets, section titles
\setbeamercolor{frametitle}{fg=C6,bg=}
\setbeamercolor{title}{fg=C6}
\setbeamercolor{subtitle}{fg=C5}
\setbeamercolor{section in toc}{fg=C6}
\setbeamercolor{itemize item}{fg=black}
\setbeamercolor{itemize subitem}{fg=C5}
\setbeamercolor{block title}{fg=Wh,bg=C6}
\setbeamercolor{block body}{bg=Wg!40,fg=Bl}
\setbeamercolor{alerted text}{fg=C4}
\setbeamercolor{example text}{fg=C3}

\setbeamertemplate{itemize item}{\color{black}$\blacktriangleright$} % first-level
\setbeamertemplate{itemize subitem}{\color{black}$\bullet$}           % second-level
\setbeamertemplate{itemize subsubitem}{\tiny$\blacksquare$}    
% Arabic with a closing parenthesis: 1) 2) 3)
\setbeamertemplate{enumerate item}{\arabic{enumi})}
\setbeamertemplate{enumerate subitem}{(\alph{enumii})}   % (a) (b) ...
\setbeamertemplate{enumerate subsubitem}{\roman{enumiii}.} % i. ii. iii.


% Change section numbering format
\setbeamertemplate{section in toc}[sections numbered]
\setbeamertemplate{subsection in toc}[subsections numbered]




% ...

 \title{Unit 2}
 \subtitle{Expectation, Normal distribution and CLT}
 \author{Ethan Levien}
 \institute{Math 50}
 \date{\today}



\begin{document}

% Title slide
\begin{frame}
  \titlepage
\end{frame}

% Outline
\begin{frame}{Outline}
  \tableofcontents
\end{frame}

% Section: Introduction
\section{Introduction}
\begin{frame}{Sample averages}
In statistics, we often have iid samples $Y_1,\dots,Y_n$. Our goal is to say something about the probability distribution $P_Y(y)$. For example, 
\begin{itemize}
\item The sample mean {\bf sample mean}
\begin{equation*}
\bar{Y} = \frac{1}{n}\sum_{i=1}^n Y_i
\end{equation*} 
\item The probability of a given outcome: 
\begin{equation*}
\bar{Y} = \frac{1}{n}\sum_{i=1}^n 1_A(Y_i) \approx P(\{Y \in A\})
\end{equation*}
\end{itemize}
In both cases we are using an {\bf sample average}. 
\end{frame}


\begin{frame}{Expectation}
\begin{itemize}
\item The sample average converges to an expectation 
\begin{equation*}
E[Y] = \sum_{y \in S}yP(\{Y = y\}) = \sum_{y \in S}y\frac{n_y}{n} = \frac{1}{n}\sum_{i=1}^nY_i
\end{equation*}
The function $E$ takes a random variable and outputs a deterministic quantity. 
\item The sample average therefore converts are random dataset into a (approximately) deterministic number which can be used to deduce attributes of our probability model. 
\end{itemize}

The central structure of classical statistics is
\begin{center}
\begin{equation*}
 \underset{\text{\tiny (from data)}}{\bar{g(Y)}} \approx \underset{}{E[f(Y)]} \underset{\text{\tiny solving equations}}{\longrightarrow} \text{Model parameters}
\end{equation*}
\end{center}


\end{frame}


\begin{frame}{Example}
We are given $n$ samples $X_i$ of a Bernoulli random variable. If we want to estimate the parameter $q$, then 
\begin{equation}
q = P(X = 1) = 0 \times P(X = 0) + 1 \times P(X=1) = E[X]
\end{equation}
Therefore 
\begin{equation}
q \approx \bar{X}. 
\end{equation}
We say that $\bar{X}$ is an {\bf estimator} of $q$ and write $\hat{q}$. In general we write $\hat{\theta}$ for an estimator of some parameter $\theta$. 


\end{frame}


\begin{frame}{Conditional expectation}
Regression models are built upon the idea of conditional expectation.
\begin{itemize}
\item  Conditional expectation is defined by
\begin{align*}
E[Y|X=x] &= \sum_{y \in S}yP(\{Y = y\}|X=x) 
\end{align*}
\item Given samples $(X_1,Y_1),\dots,(X_2,Y_2)$, we have
\begin{equation}
E[Y|X=x]  \approx  \sum_{y \in S}y\frac{P(Y = y,X=x)}{P(X=x)}\\
& = \sum_{y \in S}y\frac{n_{x,y}/n}{n_x/n} = \frac{1}{|\{i:X_i =x\}|}\sum_{\{i:X_i =x\}}X_i
\end{equation}
where $n_{x,y}$ is the number of samples where $X=x$ and $Y=y$. 
\item In Python this looks like: 
\end{itemize}



\end{frame}

\begin{frame}{Sample distribution}
How ``good'' an approximation is this? Intuitively, it will become better when $n$ becomes large, but how to we quantity 

\end{frame}



\end{document}